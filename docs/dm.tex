\documentclass{article}

\usepackage[top=3cm, bottom=3cm, left=3cm, right=3cm]{geometry}

\usepackage[french]{babel}
\usepackage[utf8]{inputenc}
\usepackage[T1]{fontenc}

\usepackage{amsmath}
\usepackage{amssymb}

\usepackage[a4paper,colorlinks,linkcolor=darkgray,citecolor=red,urlcolor=blue]{hyperref}

\usepackage{tikz}
\usetikzlibrary{patterns} % pour hachures les figures

\usepackage[]{algorithm2e}

\title{Algorithmique et programmation parallèle : DM}

\author{William \textsc{AUFORT} et Raphaël \textsc{Charrondière}}

\date{7 Décembre 2014}

\begin{document}
\maketitle

\section{Automate cellulaire}

\underline{Note :} Dans toute cette partie, on note $p^2$ le nombre de processeurs, $N$ la taille de la grille (qui contient donc $N^2$ éléments).
On supposera comme à l'habitude que $p$ divise $N$.

\subsection*{Question 1}

Pour calculer $X^{t+1}$ à partir de $X^t$ il faut appliquer la fonction $\delta$ à tous les éléments de $X^t$, soit $N^2$ fois.
Donc, pour calculer $X^t$ à partir de $X^0$, il faut répéter cette opération $t$ fois, ce qui donne $t N^2$ applications de la fonction $\delta$ nécessaires.

\subsection*{Question 2}

Nous donnons ici d'abord l'implémentation de l'automate cellulaire de la partie 3, et non pas l'implémentation de l'automate général (où les neuf voisins sont nécessaires pour la mise à jour). Nous expliquerons les différences après la description de l'algorithme. \\

L'idée est de disposer les processeurs selon une grille. Chaque processeur dispose d'un bloc de $\left( cac{N}{p} \right) ^2$ éléments.\\

Considérons un processeur $P$. Celui-ci dispose des données :
$$(\widehat{x_{i,j}})= \left\{ x_{l,m} / ki\leq l<(k+1)i,cj\leq m<(c+1)j \right\}$$

c'est a dire d'un morceau de la grille de taille $k \times c$.

Pour effectuer un calcul $\delta(\widehat{x_{i,j}})$, P a besoin en plus de $\widehat{x_{i-1,j}}$, $\widehat{x_{i+1,j}}$, $\widehat{x_{i,j-1}}$ et $\widehat{x_{i,j+1}}$.
On peut calculer l'image des éléments de $P_i$ qui ne sont pas à la frontière (les éléments en rouge sur la figure \ref{datas}), $P_i$ n'a besoin que de valeurs qui lui sont connues.
Les éléments à la frontière nécessitent quant à eux une ou deux valeurs dont disposent les processeurs voisins de $P_i$ sur la grille (les éléments en bleu sur la figure \ref{datas}) pour pouvoir être mis à jour.

\begin{figure}[!h]
  \centering
	\begin{tikzpicture}[scale = 0.5]
	  \draw[very thin, gray] (0,0) grid (8,8);
	  \fill[pattern = north east lines, pattern color = red] (2,2) -- (6,2) -- (6,6) -- (2,6) -- cycle;
	  \fill[pattern = north east lines, pattern color = blue] (0,1) -- (0,7) -- (1,7) -- (1,1) -- cycle;
	  \fill[pattern = north east lines, pattern color = blue] (1,7) -- (1,8) -- (7,8) -- (7,7) -- cycle;
	  \fill[pattern = north east lines, pattern color = blue] (7,1) -- (7,7) -- (8,7) -- (8,1) -- cycle;
	  \fill[pattern = north east lines, pattern color = blue] (1,0) -- (1,1) -- (7,1) -- (7,0) -- cycle;
	  \draw[very thick] (1,0) -- (1,8); \draw[very thick, dashed] (1,-1) -- (1,0); \draw[very thick, dashed] (1,8) -- (1,9);
	  \draw[very thick] (0,1) -- (8,1); \draw[very thick, dashed] (-1,1) -- (0,1); \draw[very thick, dashed] (8,1) -- (9,1);
	  \draw[very thick] (7,0) -- (7,8); \draw[very thick, dashed] (7,-1) -- (7,0); \draw[very thick, dashed] (7,8) -- (7,9);
	  \draw[very thick] (0,7) -- (8,7); \draw[very thick, dashed] (-1,7) -- (0,7); \draw[very thick, dashed] (8,7) -- (9,7);
	\end{tikzpicture}
  \caption{Schéma des données d'un processeur. Les données en rouge peuvent être mises à jour par le processeurs, alors que les autres nécessitent les données en bleu des processeurs voisins.}
  \label{datas}
\end{figure}

Chaque processeurs va donc recevoir deux lignes (provenant de ses voisins horizontaux) et deux colonnes (provenant de ses voisins verticaux).
Chaque processeurs doit donc également envoyer les lignes (respectivement les colonnes) supérieures et inférieures dont ont besoin les processeurs voisins horizontaux (repectivement verticaux), qui correspondent exactement aux éléments de la frontière.

Il est intéressant de remarquer que ces envois et réceptions peuvent être fait en parallèle, mais également pendant les calculs pour les éléments qui ne sont pas sur la frontière.

\paragraph{Cas général}

Dans le cas général, il y a une petite subtilité. Pour pouvoir mettre à jour les quatres éléments aux "coins", il faut disposer également d'une donnée de chaque processeurs voisin selon la diagonale. Mais cette donnée ne peut pas être transmise directement, à cause de la topologie de grille. Imaginons maintenant que l'on échange d'abord les colonnes voisines, à cette étape les voisins en haut et en bas d'un processeur connaissent ces "coins" (voir figure \ref{datas2}), il suffit donc d'échanger ces valeurs en même temps que les lignes voisines.

\begin{figure}[!h]
  \centering
	\begin{tikzpicture}[scale = 0.5]
	  \draw[very thin, gray] (0,0) grid (8,8);
	  \fill[pattern = north east lines, pattern color = red] (7,7) -- (7,8) -- (8,8) -- (8,7) -- cycle;
	  \draw[thick,->] (7.5,7.5) -- (6.5,7.5);
	  \draw[thick,->] (6.5,7.5) -- (6.5,6.5);
	  \draw (4,9) node[scale=1.5]{$P_1$};
	  \draw (4,4) node[fill=white,scale=1.5]{$P_2$};
	  \draw (9,9) node[scale=1.5]{$P_3$};
	  \draw[very thick] (1,0) -- (1,8); \draw[very thick, dashed] (1,-1) -- (1,0); \draw[very thick, dashed] (1,8) -- (1,9);
	  \draw[very thick] (0,1) -- (8,1); \draw[very thick, dashed] (-1,1) -- (0,1); \draw[very thick, dashed] (8,1) -- (9,1);
	  \draw[very thick] (7,0) -- (7,8); \draw[very thick, dashed] (7,-1) -- (7,0); \draw[very thick, dashed] (7,8) -- (7,9);
	  \draw[very thick] (0,7) -- (8,7); \draw[very thick, dashed] (-1,7) -- (0,7); \draw[very thick, dashed] (8,7) -- (9,7);
	\end{tikzpicture}
  \caption{Transmission des 4 données manquantes. La donnée (en rouge) qui intéresse $P_2$ est transmise à $P_1$ lors de la première phase où $P_3$ transmet sa colonne. $P_2$ peut donc demander cette données directement à $P_1$.}
  \label{datas2}
\end{figure}

\subsection*{Question 3}

L'algorithme final est présenté ci-dessous :

\begin{algorithm}[H]
  \KwData{Une matrice $X$}
  \KwResult{La matrice $Y = \delta^{+}\left(X\right)$}
  p $\longleftarrow \sqrt{NbProcs()}$\;
  q $\longleftarrow$ MyNum()\;
  N $\longleftarrow$ SizeOfGrid()\;
  \For{i = 1 to ($\frac{N}{p}-2$)}{
	\For{j = 1 to ($\frac{N}{p}-2$)} {
	  Mettre à jour l'élément $\widehat{x_{i,j}}$ avec les données de $P_q$.
    }
  }
  // send(les deux lignes et les deux colonnes) \;
  // receive(les deux lignes et les deux colonnes) \;		
  \For{j = 0 to ($\frac{N}{p}-1$)}{
	Mettre à jour les elements $\widehat{x_{0,j}}$ et $\widehat{x_{N-1,j}}$ avec les données reçues\;
  }
  \For{i = 1 to ($\frac{N}{p}-2$)}{
	Mettre à jour les elements $\widehat{x_{i,0}}$ et $\widehat{x_{i,N-1}}$ avec les données reçues\;
  }
 \caption{L'algorithme de la question 3}
\end{algorithm}

\subsubsection*{Complexité}

Soit $\omega$ le temps nécessaire pour effectuer une mise à jour d'un élément $x_{i,j}$.
La première partie de l'algorithme (premières mises à jour, send et receive) prend un temps $max \left( (\frac{N}{p}-2)^2\omega, 4(L + (\frac{N}{p}-2)b) \right) $.

La seconde partie prend un temps $(\frac{2N}{p} + 2(\frac{N}{p}-2))\omega$ = $(\frac{4N}{p}-4)\omega$.

D'où une complexité temporelle finale : $T(N,p) = max \left( \left(\frac{N}{p}-2\right)^2\omega, 4\left(L + \left(\frac{(N-2)b}{p}\right) \right) \right)  + \left(\frac{4N}{p}-4 \right)\omega $.

\underline{Remarque :} Quand $N$ tend vers $+\infty$, $T(N,p) \underset{N \rightarrow +\infty}{\thicksim} \left(\frac{N}{p}-2\right)^2\omega \underset{N \rightarrow +\infty}{\thicksim} \frac{N^2\omega}{p^2}$. 

\subsubsection*{Adaptation sur une grille non torique}

Si on conserve avec exactitude le principe de l'algorithme, le seul problème consistera pour les processeurs en bordure de grille, lors des envois et des réceptions.
Concrètement, si on note cette fois $P_{i,j}$ les processeurs ($(i,j) \in \left\{ 0, \dots,p-1\right\}$) selon leur place dans la grille, les processeurs $P_{0,i}$ ne peuvent pas envoyer leur ligne au processeur $P_{p-1,i}$ (même chose de $P_{i,0}$ vers $P_{i,p-1}$).
Une solution pourrait être de les faire passer à travers toute une ligne (respectivement toute une colonne), ce qui nécessiterai un temps $(p-1)(L + \frac{(n-2)b}{p})$ supplémentaire pour chaque envoi, soit un temps total de $4p(p-1)(L + \frac{(n-2)b}{p})$. Toutefois ces envois peuvent se faire en parallèle d'autres calculs. % TODO : trouver un moyen de faire cela.

\subsubsection*{Et sur une topologie en anneau ?}

%Sur une topologie en anneau la situation se complique... Les transmissions verticales devront passer par exactement $p-1$ processeurs intermédiaires avant d'arriver à destination. Toutefois, on peut réutiliser la technique vue en cours pour le broadcast généralisé (où toutes les données transitent en même temps) pour les deux transmissions verticales que doivent accomplir chaque processeurs.
Sur une topologie en anneau la répartion en grille impose clairement de faire transiter des informations par les processus, les transmissions verticales devraient passer par exactement $p-1$ processeurs intermédiaires. Cependant en attribuant des lignes complètes et consécutives de la grille on n'utilise que 2 communications directes par processeur. La quantité de données passant par chaque lien sera du même ordre de grandeur, $\frac{p-1}{p}$ fois celle d'avant, mais la latence sera diminuée de $2pL$ à $2L$.

\section{Average automata}

\subsection*{Question 4} 

Voir le fichier \texttt{average.c}.

\subsection*{Question 5}

Pour prouver que $\delta^+$ est linéaire, il suffit de montrer que pour tous $(i,j) \in \mathbb{N}$, pour tout $k \in \mathbb{R}$, 
$$(\delta^+(X+Y))_{i,j} = (\delta^+(X))_{i,j} + (\delta^+(Y))_{i,j} \quad \text{et} \quad (\delta^+(kX))_{i,j} = k(\delta^+(X))_{i,j}$$.

Sachant que $(\delta^+(X))_{i,j} = (1-p)x_{i,j} + \frac{p}{4} \left(x_{i,j+1} + x_{i,j-1} + x_{i+1,j} + x_{i-1,j}\right)$, le résultat est immédiat.

\subsubsection*{Principe de l'algorithme}

L'idée est que si on a une fonction linéaire, on peut la représenter par une matrice, mais celle-ci sera de taille $N^4$.

Quelques éléments d'explications. Ici on identifie $\mathcal{M}_N(\mathbb{R})$ et $\mathbb{R}^{N^2}$ pour représenter la matrice $X$. La matrice $X$ sous forme de colonne sera notée $\tilde{X}$. \\

$$ X = \begin{pmatrix} a & b & c \\ d & e & f \\ g & h & i \end{pmatrix} \qquad \longrightarrow \qquad \tilde{X} = \left( \begin{array}{c} a \\ b \\ c \\ d \\ e \\ f \\ g \\ h \\ i \end{array} \right)$$

On peut alors voir la fonction $\delta^+$ sous forme matricielle :
 
$$ \Delta =  \begin{pmatrix} p & \frac{1-p}{4} & \frac{1-p}{4} & \frac{1-p}{4} & 0 & 0 & \frac{1-p}{4} & 0 & 0 \\
							\frac{1-p}{4} & p & \frac{1-p}{4} & 0 & \frac{1-p}{4} & 0 & 0 & \frac{1-p}{4} & 0 \\
							\frac{1-p}{4} & \frac{1-p}{4} & p & 0 & 0 & \frac{1-p}{4} & 0 & 0 & \frac{1-p}{4} \\
							\frac{1-p}{4} & 0 & 0 & p & \frac{1-p}{4} & \frac{1-p}{4} & \frac{1-p}{4} & 0 & 0 \\
							0 & \frac{1-p}{4} & 0 & \frac{1-p}{4} & p & \frac{1-p}{4} & 0 & \frac{1-p}{4} & 0 \\
							0 & 0 & \frac{1-p}{4} & \frac{1-p}{4} & \frac{1-p}{4} & p & 0 & 0 & \frac{1-p}{4} \\
							\frac{1-p}{4} & 0 & 0 & \frac{1-p}{4} & 0 & 0 & p & \frac{1-p}{4} & \frac{1-p}{4} \\
							0 & \frac{1-p}{4} & 0 & 0 & \frac{1-p}{4} & 0 & \frac{1-p}{4} & p & \frac{1-p}{4} \\
							0 & 0 & \frac{1-p}{4} & 0 & 0 & \frac{1-p}{4} & \frac{1-p}{4} & \frac{1-p}{4} & p
			\end{pmatrix}$$

En effet : \\

$$ \begin{pmatrix} p & \frac{1-p}{4} & \frac{1-p}{4} & \frac{1-p}{4} & 0 & 0 & \frac{1-p}{4} & 0 & 0 \\
							\frac{1-p}{4} & p & \frac{1-p}{4} & 0 & \frac{1-p}{4} & 0 & 0 & \frac{1-p}{4} & 0 \\
							\frac{1-p}{4} & \frac{1-p}{4} & p & 0 & 0 & \frac{1-p}{4} & 0 & 0 & \frac{1-p}{4} \\
							\frac{1-p}{4} & 0 & 0 & p & \frac{1-p}{4} & \frac{1-p}{4} & \frac{1-p}{4} & 0 & 0 \\
							0 & \frac{1-p}{4} & 0 & \frac{1-p}{4} & p & \frac{1-p}{4} & 0 & \frac{1-p}{4} & 0 \\
							0 & 0 & \frac{1-p}{4} & \frac{1-p}{4} & \frac{1-p}{4} & p & 0 & 0 & \frac{1-p}{4} \\
							\frac{1-p}{4} & 0 & 0 & \frac{1-p}{4} & 0 & 0 & p & \frac{1-p}{4} & \frac{1-p}{4} \\
							0 & \frac{1-p}{4} & 0 & 0 & \frac{1-p}{4} & 0 & \frac{1-p}{4} & p & \frac{1-p}{4} \\
							0 & 0 & \frac{1-p}{4} & 0 & 0 & \frac{1-p}{4} & \frac{1-p}{4} & \frac{1-p}{4} & p
	\end{pmatrix} \left( \begin{array}{c} a \\ b \\ c \\ d \\ e \\ f \\ g \\ h \\ i \end{array} \right) =  
	\left( \begin{array}{c} \delta(a) \\ \delta(b) \\ \delta(c) \\ \delta(d) \\ \delta(e) \\ \delta(f) \\ \delta(g) \\ \delta(h) \\ \delta(i) \end{array} \right)  = \delta^+  \left[ \left( \begin{array}{c} a \\ b \\ c \\ d \\ e \\ f \\ g \\ h \\ i \end{array} \right) \right] $$

Nous avons donc exhibé une matrice $\Delta$ telle que $\Delta \widetilde{X^t} = \widetilde{X^{t+1}}$. Calculer $X^t$ à partir de $X_0$ revient donc à calculer $\Delta^t \widetilde{X_0}$, ce qui peut se faire en temps $O(\text{log t})$ (avec $N$ fixé) par exponentiation rapide. De plus, pour calculer les produits de matrice (ceux utiisés pour $\Delta^t$ et le produit $\Delta^+ \times X^0$), on peut utiliser l'algorithme vu en cours pour le produit matrice/matrice sur une grille de processeurs.

L'algorithme est donc le suivant : \\

\begin{algorithm}[H]
  \KwData{Une matrice $X^0$, un entier $t$}
  \KwResult{La matrice $X^t$}
  Calculer la matrice $\Delta$\;
  Calculer $\Delta^t$ via exponentiation rapide\;
  Calculer $\Delta^t \times X^0 = X^t$ \;
  Retourner $X^t$.
 \caption{L'algorithme de la question 5}
\end{algorithm}

\subsubsection*{Complexité en temps}

Dans cette question, on pose $T_{prod}(N,p)$ le temps nécessaire pour calculer le produit de deux matrices de taille $N \times N$ sur une grille de $p \times p$ processeurs. On admet que l'on dispose d'un algorithme qui s'execute en temps $T_{prod}(N,p) = max \left( pL + N^2 b, \frac{N^3}{p} \omega \right) $ (un tel algorithme a été vu en cours).

Pour calculer la matrice $\Delta$, il suffit juste d'initialiser une matrice avec $N^4$ éléments (car $\Delta \in \mathcal{M}_{N^2}(\mathbb{R})$). Cette initiatisation peut se faire de manière automatique car on a : 

$$ \Delta = 
	\left(\begin{array}{c|c|c|c}
    A_N & B_N & \dots & B_N  \\
	\hline
	B_N & A_N & \dots & B_N \\
	\hline
	\dots & \dots & \dots & B_N \\
	\hline
	B_N & \dots & \dots & A_N
	\end{array} \right)$$

$$ \text{Où :} \quad B_N = \frac{1-p}{4} \times I_N \quad \text{et} \quad A_N =
	\begin{pmatrix} p & \frac{1-p}{4} & \frac{1-p}{4} & \frac{1-p}{4} \\
					\frac{1-p}{4} & p & \frac{1-p}{4} & \frac{1-p}{4} \\
					\frac{1-p}{4} & \frac{1-p}{4} & p & \frac{1-p}{4} \\
					\frac{1-p}{4} & \frac{1-p}{4} & \frac{1-p}{4} & p
	 \end{pmatrix}  $$ \\

Pour l'exponentiation rapide, on sait qu'on effectue dans le pire des cas $\Theta(\text{log t})$ produits de matrices, soit une complexité en $\Theta(T_{prod}(N^2,p)\times \text{log t}) = \Theta \left( max \left( pL + N^4 b, \frac{N^6}{p} \omega \right) \times \text{log t} \right) $ pour le calcul de $\Delta^t$. \\

Enfin, le dernier produit matrice/vecteur se fait en temps $T_{prodVect} = max \left( pL + N^2 b, \frac{N^4}{p} \omega \right) $ (vu en cours, en utilisant $\sqrt{p}$ processeurs).

D'où une complexité temporelle finale de l'ordre de : $max \left( pL + N^4 b, \frac{N^6}{p} \omega \right) \times \text{log t} + max \left( pL + N^2 b, \frac{N^4}{p} \omega \right)$.

Remarque : le deuxième terme de la somme est ici négligeable.

\subsubsection*{Complexité en espace}

Concernant la complexité en espace, chaque processeur dispose d'un bloc de $\Delta$ de taille $\left(\frac{N^2}{p}\right)^2$, d'un bloc de $X^0$ de taille $\frac{N^2}{p}$, mais également d'un autre bloc de taille $\left(\frac{N^2}{p}\right)^2$ nécessaire au calcul de $\Delta^t$. Il ne faut pas non plus oublier que des blocs temporaires doivent être alloués pour les envois et réceptions de lignes et de colonnes durant les étapes où on effectue des multiplications. % TODO : Cours pour voir ce que celà représente comme mémoire

\subsubsection*{Comparaison dans le cas général} 

Si on regarde les complexités temporelles, 

\subsection*{Question 6}

Si le réseau de processeurs forme une clique, l'algorithme précédent change uniquement dans la manière dont les communications sont effectuées. \\

Concernant les envois de données, ceux-ci peuvent être effectués sans passer par tout le réseau, chaque envoi aura donc un coût identique de $L + \frac{N^2}{p} b$. \\

Le calcul de $\Delta$ reste inchangé, sachant qu'il ne nécessite aucune communication entre processeurs. \\

Pour calculer $\Delta^t$, on peut envoyer directement les blocs de matrices de la source à la destination. Sachant que pour calculer un bloc de taille ${\left( \frac{N}{p} \right)}^2$ il faut $2p$ blocs, le coût de cette opération est de $2p \times \left( L + {\left( \frac{N}{p} \right)}^2 \right) b$. \\

Enfin, on utilise la même idée pour calculer le dernier produit matrice/vecteur, 

\subsection*{Question 7}

L'idée est d'utiliser la matrice $Z^t$ afin d'éviter le calcul direct de $X^t$. On peut écrire $X^0$ sous la forme : 

$$ X^0 = \sum_{\underset{x^0_{i,j} \neq 0}{(i,j)}} x^0_{i,j}E_{i,j} $$

où $E_{i,j}$ est la matrice dont tous les coefficients sont nuls, excepté le coefficient en $(i,j)$ qui vaut 1. On remarquera que $Z = E_{0,0}$.

Comme $\delta^+$ est linéaire, on a :

$$ X^t = \sum_{\underset{x^0_{i,j} \neq 0}{(i,j)}} x^0_{i,j}E_{i,j}^t $$

Cherchons maintenant comment calculer $E_{i,j}^t$. Vu que l'on travaille sur une grille torique, il est clair que $E_{i,j}^t$ et $Z^t$ sont égales à une permutation près. Cette permutation constitue juste en un décalage des indices de la matrice. 
% TODO : Finir cette partie, avec le code de sparse.


\subsection*{Question 8}

Voir le fichier \texttt{sparse.c}.

\section{Réservoir thermique}

\subsection*{Question 10}


Avec une constante un point fixe est la distribution uiforme sur la grille, c'est à dire toute les cases ont la même température que la source. 

Intuitivement, si on a au moins une constante, sa valeur va être diffusée à travers la grille dans un certain voisinage, comme nous avons pu l'observer sur des exemples lors des tests de notre programme.
Cependant, donner une expression explicite de la limite semble être un peu délicat. On peut intuiter que la limite se rapprocherai d'un point fixe de la fonction $\delta^+$ (en effet, quand on est proche de la limite, on a tendance à y rester). Comme $\delta^+$ est linéaire, si elle possède un point fixe, alors en multipliant par une constante on trouve un autre point fixe. On peut tout de même représenter le système d'équations par une matrice, qui reprendrait un peu la forme de la matrice $\Delta$, dont on rappelle l'expression :
$$ \Delta = 
	\left(\begin{array}{c|c|c|c}
    A_N & B_N & \dots & B_N  \\
	\hline
	B_N & A_N & \dots & B_N \\
	\hline
	\dots & \dots & \dots & B_N \\
	\hline
	B_N & \dots & \dots & A_N
	\end{array} \right)$$

Cependant ici toutes les valeurs sont variables ici. Pour modéliser le fait que certaines valeurs soient des constantes, il suffit de remplacer la bonne ligne par une ligne de la forme $L_i = \left( 0, \dots, 0, 1, 0, \dots, 0 \right)$ (la valeur 1 étant à la ième position). Ainsi, par application de $\Delta$, cette valeur resterai inchangée.

On pourrait alors chercher un vecteur $X$ tel que $\Delta X = X$ (c'est la définition d'un point fixe de $\delta^+$).


\subsection*{Question 11}

Cette question est un peu ambigüe. L'additivité est clairement définie sur $\mathcal{M}_n(\mathbb{R})$, mais pas sur $\mathcal{M}_n(\mathbb{R} \cup \mathcal{C})$. En effet, il est difficile de définir le "type" (constant ou non) de la somme $x + \mathcal{C}(y)$. Si l'on considère que le résultat est une variable (cas 1), on perd le côté constant de $y$, alors que si on considère que le résultat est une constante (cas 2, le choix qui nous paraît le plus raisonnable), on perd le côté variable de $x$.

Mais quelque soit la convention que l'on choisisse d'adopter, l'application $\delta^+$ n'est plus linéaire. En effet, considérons deux matrices $X$ et $Y$ telles que $X$ soit nulle sauf pour $x_{0,0} = x \neq 0$, et $Y$ soit nulle sauf pour $y_{0,0} = \mathcal{C}(y) \neq \mathcal{C}(0)$.
On a clairement $\delta(x_{0,0}) = p \times x$ et $\delta(x_{0,0}) = y$.

Par contre, $\delta(x_{0,0} + y_{0,0}) = p \times (x+y)$ dans le cas 1, $x+y$ dans le cas 2. Dans tous les cas, ces deux valeurs différent de $p \times x + y$ si on suppose en plus que $p$ est non nul. Ceci prouve que les matrices $\delta^+(X) + \delta^+(Y)$ et $\delta^+(X+Y)$ différent en leur élément $(0,0)$, ce qui montre bien que $\delta^+$ n'est plus linéaire dans le cas où on rajoute des constantes.

\subsection*{Question 12}

Reprenons la question 10, on voit clairement que notre matrice est la matrice de transition d'une chaîne de Markoff or mettre une source impose l'apériodicité,
on peut donc appliquer le thèorème de Perron-Frobenius : il y a un seul point fixe qui sera atteint asymptotiquement.

% TODO : pas une simple convergence de puissance de matrices, car on ne peut modéliser les valeurs constantes... N'y réfléchir que si tout est fait

\subsection*{Question 13}

Voir le fichier \texttt{constants.c}, très inspiré de \texttt{average.c}.

\subsection*{Question 14}

Voir les fichiers \texttt{gfx.h} et \texttt{gfx.c}.

\end{document}
