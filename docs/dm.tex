\documentclass{article}

\usepackage[top=3cm, bottom=3cm, left=3cm, right=3cm]{geometry}

\usepackage[french]{babel}
\usepackage[utf8]{inputenc}
\usepackage[T1]{fontenc}

\usepackage{amsmath}
\usepackage{amssymb}

\usepackage[a4paper,colorlinks,linkcolor=darkgray,citecolor=red,urlcolor=blue]{hyperref}

\title{Algorithmique et programmation parallèle : DM}

\newcommand{\X}[1]{{X}^{ #1 }}

\author{William \textsc{AUFORT} et Raphaël \textsc{Charrondière}}

\date{7 Décembre 2014}

\begin{document}
\maketitle

\section*{Question 1}

Pour calculer $\X{t+1}$ à partir de $\X{t}$ il faut appliquer la fonction $\delta$ à tous les éléments de $\X{t}$, soit $N^2$ fois.
Donc, pour calculer $\X{t}$ à partir de $\X{0}$, il faut répéter cette opération $t$ fois, ce qui donne $t N^2$ applications de la fonction $\delta$ nécessaires.

\section*{Question 2}

Nous donnons ici l'implémentation de l'automate cellulaire de la partie 3, et non pas l'implémentation de l'automate général (où les neuf voisins sont nécessaires pour la mise à jour). On expliquera néanmoins les différences après la description de l'algorithme. \\

Soient $p^2$ le nombre de processus disponibles, N la dimension de la grille. On suppose que $p$ divise $n$.
L'idée est de donner à chaque processus de la grille un bloc de $\left( \frac{N}{p} \right) ^2$ éléments selon la même topologie que celle de la grille de processus. \\

Considérons un processus $P_i$.
Pour effectuer un calcul $\delta(x)$, on a besoin des cases au dessus, en dessous, à gauche et à droite.
On peut calculer l'image de certains éléments de $P_i$ (ceux qui ne sont pas à la frontière) uniquement avec les valeurs connues de $P_i$. 
Les éléments à la frontière nécessite une ou deux valeurs dont dispose les processus voisins de $P_i$ sur la grille.

% TODO : figure Tikz comme celle que l'on a fait au tableau vendredi

Chaque processus $P_i$ va donc recevoir deux lignes (provenant de ses voisins horizontaux) et deux colonnes (provenant de ses voisins latéraux), et lui-même doit envoyer ses lignes et ses colonnes.

Il est intéressant de remarquer que ces envois et réceptions peuvent être fait en parallèle des calculs pour les éléments qui ne sont pas sur la frontière.

L'algorithme final est donc le suivant :

% TODO : Algorithme pour P_i
	

\end{document}
